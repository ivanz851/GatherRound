\documentclass[a4paper,12pt]{extarticle}
\usepackage[utf8]{inputenc}
\usepackage[english,russian]{babel}
\usepackage{csquotes}
\usepackage{geometry}
\usepackage[T1]{fontenc}
\usepackage[utf8]{inputenc}
\usepackage[english,russian]{babel}
\usepackage{amsmath}
\usepackage{amsthm}
\usepackage{amssymb}
\usepackage{fancyhdr}
\usepackage{setspace}
\usepackage{graphicx}
\usepackage{colortbl}
\usepackage{tikz}
\usepackage{pgf}
\usepackage{subcaption}
\usepackage{listings}
\usepackage{indentfirst}
\usepackage{listings}
\usepackage{xcolor}
\usepackage{graphicx}
\usepackage{enumitem}
\usepackage{titlesec}

\usepackage[
backend=biber,
style=numeric,
maxbibnames=99
]{biblatex}
\addbibresource{refs.bib}
\usepackage[colorlinks,citecolor=blue,linkcolor=blue,bookmarks=false,hypertexnames=true, urlcolor=blue]{hyperref} 
\usepackage{indentfirst}
\usepackage{mathtools}
\usepackage{booktabs}
\usepackage[flushleft]{threeparttable}
\usepackage{tablefootnote}

\usepackage{chngcntr} % нумерация графиков и таблиц по секциям
\counterwithin{table}{section}
\counterwithin{figure}{section}

\graphicspath{{graphics/}}%путь к рисункам

\makeatletter
% \renewcommand{\@biblabel}[1]{#1.} % Заменяем библиографию с квадратных скобок на точку:
\makeatother

\geometry{left=2.5cm}% левое поле
\geometry{right=1.0cm}% правое поле
\geometry{top=2.0cm}% верхнее поле
\geometry{bottom=2.0cm}% нижнее поле
\setlength{\parindent}{1.25cm}
\renewcommand{\baselinestretch}{1.5} % междустрочный интервал


\newcommand{\bibref}[3]{\hyperlink{#1}{#2 (#3)}} % biblabel, authors, year
\addto\captionsrussian{\def\refname{Список литературы (или источников)}} 

\renewcommand{\theenumi}{\arabic{enumi}}% Меняем везде перечисления на цифра.цифра
\renewcommand{\labelenumi}{\arabic{enumi}}% Меняем везде перечисления на цифра.цифра
\renewcommand{\theenumii}{.\arabic{enumii}}% Меняем везде перечисления на цифра.цифра
\renewcommand{\labelenumii}{\arabic{enumi}.\arabic{enumii}.}% Меняем везде перечисления на цифра.цифра
\renewcommand{\theenumiii}{.\arabic{enumiii}}% Меняем везде перечисления на цифра.цифра
\renewcommand{\labelenumiii}{\arabic{enumi}.\arabic{enumii}.\arabic{enumiii}.}% Меняем везде перечисления на цифра.цифра

% Определение цветов
\definecolor{codegreen}{rgb}{0,0.6,0}
\definecolor{codegray}{rgb}{0.5,0.5,0.5}
\definecolor{codepurple}{rgb}{0.58,0,0.82}
\definecolor{backcolour}{rgb}{0.95,0.95,0.92}

% Определение стиля листинга кода
\lstdefinestyle{mystyle}{
    backgroundcolor=\color{backcolour},   
    commentstyle=\color{codegreen},
    keywordstyle=\color{magenta},
    numberstyle=\tiny\color{codegray},
    stringstyle=\color{codepurple},
    basicstyle=\ttfamily\footnotesize,
    breakatwhitespace=false,         
    breaklines=true,                 
    captionpos=b,                    
    keepspaces=true,                 
    numbers=left,                    
    numbersep=5pt,                  
    showspaces=false,                
    showstringspaces=false,
    showtabs=false,                  
    tabsize=2
}

% Настройка стиля листинга кода
\lstset{style=mystyle}

\titleformat{\section}[block]{\normalfont\Large\bfseries\centering}{\thesection}{1em}{}

\title{Заявка на инициативный проект \textit{<<Gather Round>>}}
\date{}

\begin{document}

\newpage
\setcounter{page}{1}
{
  \hypersetup{linkcolor=black}
}

\maketitle

\section{Аннотация}  
\noindent "<Gather Round"> — приложение, созданное для упрощения организации встреч компаний друзей на культурных мероприятиях. 
Идея создания приложения возникла из наблюдения за сложностями, с которыми сталкиваются компании людей при выборе события и места встречи, учитывая разбросанность по городу. 
Часто процесс организации затягивается, в том числе из-за того, что некоторые из участников тратит значительно больше времени на дорогу, чем другие.

\section{Цель проекта}
\noindent Цель разработки "<Gather Round"> — предоставить пользователям инструмент, который упростит процесс организации совместных мероприятий, сделает его быстрым и удобным, а также повысит посещаемость культурных событий.

\section{Задачи}
\noindent 
- Реализовать базовую структуру приложения. 
\\
- Реализовать алгоритма построения оптимальных маршрутов между станциями.
\\
- Реализовать взаимодействие пользователя с приложением через консоль.
\\
- Изучить JavaScript и фреймворк Vue.js.
\\
- Реализовать интерактивной карты московского метро для удобства выбора станций и отображения маршрутов.
\\
- Разработать и внедрть дизайна в приложение.
\\
- Интеграция с API Яндекс.Карт и портала открытых данных Правительства Москвы.
\\
- Реализовать поиск и построение маршрута от местоположения пользователя до ближайшей станции метро и от станции назначения до места проведения события 
(с использованием API Яндекс.Карт).
\\
- Тестирование и отладка функционала.
\\
- Публикация приложения в RuStore.

\section{Планируемые результаты} 
Реализованное приложение, доступное для скачивания в RuStore.


\end{document}
