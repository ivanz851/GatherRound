\documentclass[a4paper,12pt]{extarticle}
\usepackage[utf8]{inputenc}
\usepackage[english,russian]{babel}
\usepackage{csquotes}
\usepackage{geometry}
\usepackage[T1]{fontenc}
\usepackage[utf8]{inputenc}
\usepackage[english,russian]{babel}
\usepackage{amsmath}
\usepackage{amsthm}
\usepackage{amssymb}
\usepackage{fancyhdr}
\usepackage{setspace}
\usepackage{graphicx}
\usepackage{colortbl}
\usepackage{tikz}
\usepackage{pgf}
\usepackage{subcaption}
\usepackage{listings}
\usepackage{indentfirst}
\usepackage{listings}
\usepackage{xcolor}
\usepackage{graphicx}
\usepackage{enumitem}
\usepackage{titlesec}
\usepackage{cmap}

\usepackage[
backend=biber,
style=numeric,
maxbibnames=99
]{biblatex}
\addbibresource{refs.bib}
\usepackage[colorlinks,citecolor=blue,linkcolor=blue,bookmarks=false,hypertexnames=true, urlcolor=blue]{hyperref} 
\usepackage{indentfirst}
\usepackage{mathtools}
\usepackage{booktabs}
\usepackage[flushleft]{threeparttable}
\usepackage{tablefootnote}

\usepackage{chngcntr} % нумерация графиков и таблиц по секциям
\counterwithin{table}{section}
\counterwithin{figure}{section}

\graphicspath{{graphics/}}%путь к рисункам

\makeatletter
% \renewcommand{\@biblabel}[1]{#1.} % Заменяем библиографию с квадратных скобок на точку:
\makeatother

\geometry{left=2.5cm}% левое поле
\geometry{right=1.0cm}% правое поле
\geometry{top=2.0cm}% верхнее поле
\geometry{bottom=2.0cm}% нижнее поле
\setlength{\parindent}{1.25cm}
\renewcommand{\baselinestretch}{1.5} % междустрочный интервал


\newcommand{\bibref}[3]{\hyperlink{#1}{#2 (#3)}} % biblabel, authors, year
\addto\captionsrussian{\def\refname{Список литературы (или источников)}} 

\renewcommand{\theenumi}{\arabic{enumi}}% Меняем везде перечисления на цифра.цифра
\renewcommand{\labelenumi}{\arabic{enumi}}% Меняем везде перечисления на цифра.цифра
\renewcommand{\theenumii}{.\arabic{enumii}}% Меняем везде перечисления на цифра.цифра
\renewcommand{\labelenumii}{\arabic{enumi}.\arabic{enumii}.}% Меняем везде перечисления на цифра.цифра
\renewcommand{\theenumiii}{.\arabic{enumiii}}% Меняем везде перечисления на цифра.цифра
\renewcommand{\labelenumiii}{\arabic{enumi}.\arabic{enumii}.\arabic{enumiii}.}% Меняем везде перечисления на цифра.цифра

% Определение цветов
\definecolor{codegreen}{rgb}{0,0.6,0}
\definecolor{codegray}{rgb}{0.5,0.5,0.5}
\definecolor{codepurple}{rgb}{0.58,0,0.82}
\definecolor{backcolour}{rgb}{0.95,0.95,0.92}

% Определение стиля листинга кода
\lstdefinestyle{mystyle}{
    backgroundcolor=\color{backcolour},   
    commentstyle=\color{codegreen},
    keywordstyle=\color{magenta},
    numberstyle=\tiny\color{codegray},
    stringstyle=\color{codepurple},
    basicstyle=\ttfamily\footnotesize,
    breakatwhitespace=false,         
    breaklines=true,                 
    captionpos=b,                    
    keepspaces=true,                 
    numbers=left,                    
    numbersep=5pt,                  
    showspaces=false,                
    showstringspaces=false,
    showtabs=false,                  
    tabsize=2
}

% Настройка стиля листинга кода
\lstset{style=mystyle}

\titleformat{\section}[block]{\normalfont\Large\bfseries\centering}{\thesection}{1em}{}

\title{Требования к проекту \textit{<<Gather Round>>}}
\date{}

\begin{document}

\newpage
\setcounter{page}{1}
{
  \hypersetup{linkcolor=black}
}

\maketitle


\section{Цель}
\subsection{Обзор проекта} 
\noindent
"<Gather Round"> — приложение, созданное для упрощения организации встреч компаний друзей на культурных мероприятиях. 
Идея создания приложения возникла из наблюдения за сложностями, с которыми сталкиваются компании людей при выборе события и места встречи, учитывая разбросанность по городу. 
Часто процесс организации затягивается, в том числе из-за того, что некоторые из участников тратит значительно больше времени на дорогу, чем другие.
\subsubsection{Цель продукта} 
\noindent
Цель разработки "<Gather Round"> — предоставить пользователям инструмент, который упростит процесс организации совместных мероприятий, сделает его быстрым и удобным, а также повысит посещаемость культурных событий.

\section{Потребительская ценность}
\subsection{Целевая аудитория} 
\noindent
Все возрастные категории, поскольку приложение будет предоставлять инфорацию о событиях различной направленности.
\subsection{Особенности отрасли:}
\noindent
Рост интереса к локальным культурным мероприятиям.
\\
Широкое распространение мобильных устройств и приложений для планирования досуга.

\subsection{Рыночные тренды:}
\noindent
Увеличение числа пользователей, предпочитающих мобильные приложения для организации времени.
\\
Спрос на персонализированные сервисы, учитывающие индивидуальные потребности.

\subsection{Ценностные предложения:}
\noindent
\textbf{Оптимизация места встречи:} \\
Равное время в пути для всех участников.
\\
Экономия времени и повышение комфорта при планировании встреч.
\\
\textbf{Единый сервис для поиска событий:} \\
Интеграция с API KudaGo и Портала открытых данных Правительства Москвы для получения актуальной информации о событиях.
\\
\textbf{Простота и удобство использования:} \\
Интуитивно понятный интерфейс.
\\
Быстрое взаимодействие с приложением без лишних действий.

\subsection{Рисковые гипотезы} 
\noindent
\textbf{Низкий уровень доверия к новым приложениям:} \\
Пользователи могут опасаться предоставлять данные о своем местоположении.
\\
\textbf{Технические ограничения:} \\
Возможны сложности с интеграцией API Яндекс Карт. 

\section{Программное решение}
\subsection{Основные свойства продукта:}
\noindent
\textbf{Разработка:} 
\newline
ОС: Android
\newline
Бэкенд: Kotlin
\newline
Фронтенд: JavaScript с использованием фреймворка Vue.js
\newline
\textbf{Интеграции:}
\newline
API KudaGo и Портала открытых данных Правительства Москвы для получения актуальных данных о культурных 
событиях.
\newline
API Яндекс Карт для построения оптимальных маршрутов.
\newline
\textbf{Функционал:}
% \newline
% Поиск и выбор событий по категориям.
\newline
Добавление участников встречи и расчет оптимального места встречи.
\newline
Отображение маршрутов для каждого участника с указанием 
времени в пути.
\newline
\textbf{Дизайн:}
\newline
Интуитивно понятный интерфейс.
\newline
Быстрая навигация по приложению.
\newline
% \textbf{Дополнительные возможности:}
% \newline
% Сохранение избранных событий.
% \newline
% Уведомления о предстоящих встречах.

\subsection{Этапы разработки} 
\noindent
1). Настройка инструмента для измерения покрытия кода на Kotlin тестами.
\\
2). Подача заявки и получение API ключа портала открытых данных  правительства Москвы.
\\
3). Реализация класса для представления взвешенного графа.
\\
4). Реализация и покрытие тестами алгоритма построения 
кратчайших путей в взвешенном графе.
\\
5). Реализация класса для хранения информации о станциях метро.
\\
6). Получение необходимых данных о станциях московского метро (внутри кольцевой линии) и времени в пути между ними.
\\
Парсинг и сохранение данныx для использования программой.
\\
7). Реализация и тестирование функции построения оптимального маршрута между станциями метро.
\\
8). Реализация класса для хранения информации о мероприятиях.
\\
9). Получение данные о мероприятиях и станциях метро, рядом с которыми находятся места их проведения (данные получать вручную из открытого доступа).
\\
10). Реализация и тестирование алгоритма нахождения мероприятия, время в дороге до которого у всех участников примерно одинаково.
\\
11). Реализация и тестирование класса консольного взаимодействия с пользователем.
\\
12). Реализация первой версии приложения и внедрение взаимодействия с пользователем через текстовый интерфейс в приложении
\\
13). Подготовка первой версии приложения к публикации в RuStore:
\\
- создание иконки приложения;
\\
- создание описания предназначения и функций приложения;
\\
- поготовка снимков экрана из приложения;
\\
- заполнение и подача заявки на публикацию
\\
14). Прохождение модерации и релиз в RuStore
\\
15). Подача заявки и получение API ключа сайта KudaGo.ru
\\
16). Получение необходимых данных о станциях московского метро (всех) и времени в пути между ними.
\\
Парсинг и сохранение данных для использования программой.
\\
17). Реализация и внедрение в приложение интерактивной карты метро (используется фреймворк Vue.js).
\\
18). Подготовка второй версии приложения к публикации в RuStore.
\\
19). Прохождение модерации и публикация второй версии приложения в RuStore.
\\
20). Настройка получения данных о мероприятиях и местах их проведения с сайти KudaGo.ru.
\\
21). Подготовка третьей версии приложения к публикации в RuStore.
\\
22). Прохождение модерации и публикация третьей версии приложения в RuStore.
\\
23). Подключение модуля "Метро" из API Ядекс.Карт.
\\
24). Реализация и внедрение в приложение отображения маршрута от финишной станции метро к месту проведения мероприятия.
\\
25). Подготовка четвертой версии приложения к публикации в RuStore.
\\
26). Прохождение модерации и публикация четвертой версии приложения в RuStore.


\subsection{Критерии запуска продукта:}
\noindent
Успешное прохождение локального тестирования.

\section{Реализация}
\subsection{План развития после запуска} 
\noindent
\textbf{При достижении целей:}
\newline
Расширение функционала: разработка web-версии для доступа с устройств на IOS и Windows.
\newline
Масштабирование: добавление других российских городов, в которых есть метрополитен.
\newline
\textbf{Если цели не будут достигнуты:}
\newline
Анализ: изучение отзывов пользователей для выявления причин.
\newline
Корректировка стратегии: изменение и улучшение функционала.

\end{document}
